%% start of file `template-zh.tex'.
%% Copyright 2006-2012 Xavier Danaux (xdanaux@gmail.com).
%
% This work may be distributed and/or modified under the
% conditions of the LaTeX Project Public License version 1.3c,
% available at http://www.latex-project.org/lppl/.


\documentclass[11pt,a4paper,sans]{moderncv}   % possible options include font size ('10pt', '11pt' and '12pt'), paper size ('a4paper', 'letterpaper', 'a5paper', 'legalpaper', 'executivepaper' and 'landscape') and font family ('sans' and 'roman')
\usepackage{zhfontcfg}

% moderncv 主题
\moderncvstyle{classic}                        % 选项参数是 ‘casual’, ‘classic’, ‘oldstyle’ 和 ’banking’
\moderncvcolor{blue}                          % 选项参数是 ‘blue’ (默认)、‘orange’、‘green’、‘red’、‘purple’ 和 ‘grey’
%\nopagenumbers{}                             % 消除注释以取消自动页码生成功能

% 调整页面出血
\usepackage[scale=0.75]{geometry}
%\setlength{\hintscolumnwidth}{3cm}           % 如果你希望改变日期栏的宽度
\CJKsetecglue{}

% 个人信息
\firstname{孙}
\familyname{滨}
\title{求职信}                      % 可选项、如不需要可删除本行
%\address{街道及门牌号}{邮编及城市}             % 可选项、如不需要可删除本行
\mobile{18653809933}                         % 可选项、如不需要可删除本行
%\phone{0538-5631332}                          % 可选项、如不需要可删除本行
%\fax{+3~(456)~789~012}                            % 可选项、如不需要可删除本行
\email{sd44sd44@yeah.net}                    % 可选项、如不需要可删除本行
\homepage{sd44.is-programmer.com}                  % 可选项、如不需要可删除本行
\photo[64pt][0.4pt]{picture}                  % ‘64pt’是图片必须压缩至的高度、‘0.4pt‘是图片边框的宽度 (如不需要可调节至0pt)、’picture‘ 是图片文件的名字;可选项、如不需要可删除本行
%\quote{引言(可选项)}                           % 可选项、如不需要可删除本行

% 显示索引号;仅用于在简历中使用了引言
%\makeatletter
%\renewcommand*{\bibliographyitemlabel}{\@biblabel{\arabic{enumiv}}}
%\makeatother

% 分类索引
%\usepackage{multibib}
%\newcites{book,misc}{{Books},{Others}}
%----------------------------------------------------------------------------------
%            内容
%----------------------------------------------------------------------------------
\begin{document}
\maketitle

\section{个人简历}
\cvline{姓名}{孙滨}
\cvline{性别}{男}
\cvline{出生日期}{1983/11/28}
\cvline{学历}{高中}

%\section{教育背景}
%\cventry{年 -- 年}{学位}{院校}{城市}{\textit{成绩}}{说明}
%\cventry{年 -- 年}{学位}{院校}{城市}{\textit{成绩}}{说明}

%\section{毕业论文}
%\cvitem{题目}{\emph{题目}}
%\cvitem{导师}{导师}
%\cvitem{说明}{\small 论文简介}

\section{工作背景}
\cventry{2002年-2003年}{售后}{宁阳县紫光电脑服务中心}{宁阳县}{}{全面负责紫光
电脑于宁样的售后服务、维修工作}
\cventry{2004年-2007年}{业代}{泰安市北斗酒水}{泰安市}{}{泰山特曲、五岳独尊酒水区域代理商处
酒水推销}
\cventry{2007年-2012年}{业代}{济南五岳品牌有限公司}{济南市}{}{公司为泰山酒业集团
    驻济南分公司\newline{}本人负责济南西北部地区商超、酒店、团购业务}

%\section{语言技能}
%\cvitemwithcomment{语言 1}{水平}{评价}
%\cvitemwithcomment{语言 2}{水平}{评价}
%\cvitemwithcomment{语言 3}{水平}{评价}
%

\section{ 电脑技能}
\cvlanguage{OS}{\mbox{Linux, Freebsd, Windows}}{}
\cvlanguage{programming}{\textsc{C, C++, ASM}}{}
\cvlanguage{scripting}{Shell, Python}{早期学习Python,现遗忘较多,可在短期内重新学习运用}
\cvlanguage{editor}{Vim}{}
\begin{itemize}%
    \item \small 接触UNIX/Linux系统多年,熟悉系统应用与操作。
    \item \small 了解UNIX/Linux下API,pthread多线程编程与socket套接字编程。
    \item \small 熟练掌握C语言,了解汇编32位,64位操作命令。
\end{itemize}

%\section{计算机技能}
%\cvdoubleitem{类别 1}{XXX, YYY, ZZZ}{类别 4}{XXX, YYY, ZZZ}
%\cvdoubleitem{类别 2}{XXX, YYY, ZZZ}{类别 5}{XXX, YYY, ZZZ}
%\cvdoubleitem{类别 3}{XXX, YYY, ZZZ}{类别 6}{XXX, YYY, ZZZ}

%\section{个人兴趣}
%\cvitem{爱好 1}{\small 说明}
%\cvitem{爱好 2}{\small 说明}
%\cvitem{爱好 3}{\small 说明}

\section{个人成绩}
\cventry{2010年10月}{开始自学编程}{}{}{}{}
\cventry{\hbox{2011年05月}}{软考初级程序员}{}{}{总分\textit{127}分}{上午题\textit{62}分,下午题\textit{65}分}
\cventry{2011年11月}{软考中级软件设计师}{}{}{总分\textit{119}分}{上午题\textit{55}分,下午题\textit{64}分}
%\cventry{年 -- 年}{学位}{院校}{城市}{\textit{成绩}}{说明}

%\section{其他 1}
%\cvlistitem{项目 1}
%\cvlistitem{项目 2}
%\cvlistitem{项目 3}

\renewcommand{\listitemsymbol}{-}             % 改变列表符号

\section{其他 2}
\cvlistdoubleitem{项目 1}{项目 4}
\cvlistdoubleitem{项目 2}{项目 5\cite{book1}}
\cvlistdoubleitem{项目 3}{}

% 来自BibTeX文件但不使用multibib包的出版物
%\renewcommand*{\bibliographyitemlabel}{\@biblabel{\arabic{enumiv}}}% BibTeX的数字标签
\nocite{*}
\bibliographystyle{plain}
\bibliography{publications}                    % 'publications' 是BibTeX文件的文件名

% 来自BibTeX文件并使用multibib包的出版物
%\section{出版物}
%\nocitebook{book1,book2}
%\bibliographystylebook{plain}
%\bibliographybook{publications}               % 'publications' 是BibTeX文件的文件名
%\nocitemisc{misc1,misc2,misc3}
%\bibliographystylemisc{plain}
%\bibliographymisc{publications}               % 'publications' 是BibTeX文件的文件名

\end{document}


%% 文件结尾 `template-zh.tex'.
